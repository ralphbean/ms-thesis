\documentclass{elsart1p}
%\documentclass[11pt]{report}
%\usepackage{amssymb}
%\usepackage{multirow}

\begin{document}

\begin{frontmatter}

\author{Ralph Bean - Thesis Pre-proposal}

\end{frontmatter}

%\textbf{Ralph Bean -- Thesis Pre-proposal}

Modelling data collected from the primate visual cortex has served to advance
both a biological understanding of the brain as well as engineering techniques
for smarter vision systems.  Most well understood are the mechanisms for simple
feature extraction by contrast and edge detection.  Less well understood are
the higher order mechanisms for isolated motion and optic flow.  For my 
Master's thesis I plan to work with Dr. Gaborski on Modelling MST 
Receptive Fields.

A brief background:  receptive fields are regions of the retina, that when
presented with particular stimuli, have been correlated with the excitation of 
particular cortical neurons.  Examples include the fine-resolution
contrast-sensitive receptive fields modeled by a difference of gaussians
filter, the less-fine-resolution edge detection receptive fields modeled by
gabor filters, and some MT receptive fields modeled by many-dimensional
gaussian models.

Neurons in MST have been found to respond to particular patterns of optic
flow.  Generally, optic flows are broad patterns of motion across the entire
or almost entire visual field.  They are individually characterized
by what type of self-motion would produce such a pattern of visual motion.  For
instance, outward radial motion across the entire visual scene is almost
exclusively the result of forward self-motion in the environment.  Uniform
leftward motion indicates rightward strafing or clockwise turning.  

It is worth noting the biological context.
The human brain does not compute sequentially but massively in parallel and
the interaction amongst brain regions cannot be downplayed.  Nonetheless, 
attempts to mathematically generalize the behavior of neurons in simpler 
brain regions have been successful and the development of more descriptive MST 
RF models will be useful; useful for the engineering
of autonomous robots and other non-linear control problems and 
useful to psychologists wishing to understand a brain region distinguished 
by its early role in the more phenomenal sensory-motor loop.

We have access to a set of neuroscience data collected by Dr. Duffy and his
team at the University of Rochester.  Adult Rhesus Macaques were presented
with simulated optic flow patterns on a viewing screen while recordings
were taken of individual neurons in MST.  There are three different classes of
data available, each with different stimuli but with recordings from the same
neurons for each.  Full optic flow data:  the screen is divided up into nine 
equal subregions and each region contains a motion pattern in a particular 
direction.  In concert they correspond with 'natural' patterns of optic flow.
Singles data:  the entire screen is without stimulus except for a particular
subregion with motion in a particular direction.  These stimuli excite
different MST neurons to various degrees but generally to a much lesser degree
than full optic flow.  Doubles data:  Similar to singles but with two regions
of simultaneous stimulus in varying directions.
The task can be summed up: \textit{given singles and doubles data, can we 
predict optic flow responses}?

Another student in the department has been working on this project for the
past year and has published.  However, there is some room for improvement;
some neurons' behavior was modelled more aptly than others.  For my thesis,
I plan to investigate some non-traditional models of the antecedent MT in
the literature and attempt to apply them to MST data.  Specifically,
non-uniform summation of MT outputs may be able to maintain the good fits
of previous work in the department while closing the gap on the few more 
poorly modeled neurons.
Furthermore, I hope to investigate a possible explanation of MST as a
competition between fourier components which has been published with 
respect to MT.

\vspace{20pt}

%Ralph Bean

\vspace{30pt}
%Roger Gaborski
\end{document}
