\documentclass{elsart1p}
\usepackage{amsmath, amssymb, graphicx}
\begin{document}

Here we begin the work of deriving an expression of contraints on the
parameters of a spiking neural network in terms of desired Lyapunov Exponents.

For book keeping, important equations and theorems
from the previous survey chapter will here be numbered.

Recall that the Lyapunov Exponenets are defined as:
\begin{eqnarray}
L_{k} = e^{h_{k}} = \lim \limits_{n \to \infty} (r_{k}^{n})^{\frac{1}{n}}
\end{eqnarray}
% TODO check the following
and can be alternatively expressed as:
\begin{eqnarray}
L_{k} = \sqrt{\lambda_{k}}
\end{eqnarray}
where $\lambda_{k}$ is the $k$th eigenvalue of the matrix $J_{m}J_{m}^{T}$.

%TODO matrix numbering latex
$J$ is the jacobian of $f$ given by \textit{(TODO, matrix numbering in \LaTeX)}:
\[ J = \left( \begin{array}{cccc}
\frac{\partial f_{1}}{\partial x_{1}} &.
\frac{\partial f_{1}}{\partial x_{2}} &.
\cdots &
\frac{\partial f_{1}}{\partial x_{n}} \\
\frac{\partial f_{2}}{\partial x_{1}} &.
\frac{\partial f_{2}}{\partial x_{2}} &.
 &
\frac{\partial f_{2}}{\partial x_{n}} \\
\vdots & & \ddots & \vdots \\
\frac{\partial f_{n}}{\partial x_{1}} &.
\frac{\partial f_{n}}{\partial x_{2}} &.
\cdots &
\frac{\partial f_{n}}{\partial x_{n}} \end{array} \right)\].

$J_{m}$ is the jacobian of the $m$th iterate of $f$, given by:

\[ J_{m} = \left( \begin{array}{cccc}
\frac{\partial (f_{1}^{m})}{\partial x_{1}} &.
\frac{\partial (f_{1}^{m})}{\partial x_{2}} &.
\cdots &
\frac{\partial (f_{1}^{m})}{\partial x_{n}} \\
\frac{\partial (f_{2}^{m})}{\partial x_{1}} &.
\frac{\partial (f_{2}^{m})}{\partial x_{2}} &.
 &
\frac{\partial (f_{2}^{m})}{\partial x_{n}} \\
\vdots & & \ddots & \vdots \\
\frac{\partial (f_{n}^{m})}{\partial x_{1}} &.
\frac{\partial (f_{n}^{m})}{\partial x_{2}} &.
\cdots &
\frac{\partial (f_{n}^{m})}{\partial x_{n}} \end{array} \right)\].

where $f_{i}^{m}$ does not denote the $m$th power of $f$ but the $m$th 
composition of $f$.

Recall that the derivative of functions under composition is given
by the chain rule from calculus.
\begin{eqnarray}
(f^{m}(x))^{(\prime)} = f^{(\prime)}(f^{m-1}(x)) f^{(\prime)}(f^{m-2}(x)) \cdots f^{(\prime)}(f^{2}(x))  f^{(\prime)}(x) 
\end{eqnarray}

\textbf{Remark:} $J_{m}J_{m}^{T}$ is necessarily symmetric and therefore orthogonally
diagonalizable by the Spectral Theorem from linear algebra.  Consequently,
there exists a matrix $Q$ whose columns are the orthonormal eigenvectors of
$J_{m}J_{m}^{T}$ and a matrix $D$ whose diagonal entries are the eigenvalues
of $J_{m}J_{m}^{T}$ such that
\begin{eqnarray}
Q^{T}(J_{m}J_{m}^{T})Q = D
\end{eqnarray}

\subsection{Example:  Lyapunov Exponents of the Quadratic Maps}
The quadratic map is a family of one-dimensional functions of the form 
\begin{eqnarray}
f_{\mu}(x) = \mu x (1-x)
\end{eqnarray}
 where $\mu$ is a real-valued parameter.
$f_{\mu}(x)$ is known to be chaotic on the interval $[0,1]$ for $\mu=4$ and
exhibits the period-doubling route to chaos from $\mu=2+\sqrt{5}$ to $\mu=4$.

\textbf{TODO} cobweb diagram of the logistic map for clarity

\textbf{TODO} bifurcation diagram of the logistic map for clarity

Discovering an expression for the lyapunov exponenets in terms of the 
parameter
will be relatively simple since both the phase space and the parameter space
are each one-dimensional.  We have only one possible lyapunov number, (there
are as many as there are dimensions of the phase space) so our Jacobian
will be only a $1 \times 1$ matrix.  The first derivative $f_{\mu}^{(\prime)}$ is given by:
\begin{eqnarray}
f_{\mu}^{(\prime)} = \mu - \mu x
\end{eqnarray}

\textbf{TODO} retool the below using the chain rule from calculus.

\begin{eqnarray*}
J_{1} &=& \frac{ \partial }{\partial x} (\mu x (1 - x)) \\
      &=& \mu - 2\mu x \\
J_{2} &=& \frac{ \partial }{\partial x} (\mu (\mu x (1-x)) (1 - (\mu x (1-x)))) \\
      &=& \mu - 2\mu x \\
J_{3} &=& \frac{ \partial }{\partial x} (\mu (\mu (\mu x (1-x)) (1-(\mu x (1-x)))) (1 - (\mu (\mu x (1-x)) (1-(\mu x (1-x))))) \\
      &=& \mu - 2\mu x \\
\end{eqnarray*}



See \ref{eqn:foo1}

\begin{def}

foofoo

\label{def:lyap1}
\end{def}

See \ref{def:lyap1}

\begin{thm}
The Lyapunov Exponents are given by the square root of the eigenvalues of $JJ^{T}$.
\label{thm:lyap}
\end{thm}

See \ref{thm:lyap}

\begin{thm}
{foo0u}[foo1]
foo2
\label{thm:ortho}
\end{thm}

See \ref{thm:ortho}

\begin{thm}
{foo0u}[foo1]
foo2
\label{thm:spectral}
\end{thm}

See \ref{thm:spectral}



\end{document}
