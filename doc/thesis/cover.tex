\documentstyle{letter}
\begin{document}
\begin{letter}

\textbf{M.S. Thesis Submission for the Department Computer Science}

\textit{Title:}
Vibrational Control of Chaos in Artificial Neural Networks

\textit{Author:}
Ralph Bean


\textit{Abstract:}
Neural networks with chaotic baseline behavior are interesting for
their experimental bases in both biological relevancy and engineering
applicability.  In the engineering case, the literature still lacks a
robust study of the interrelationship between particular chaotic baseline
network dynamics and ``online" or ``driving" inputs.  We ask the question,
for a particular neural network with chaotic baseline
behaviour, what periodic inputs of minimal magnitude have a
    stabilizing effect on network dynamics?  A genetic algorithm is
    developed for the task.  A systematic comparison of different
genetic operators is carried out where each operator-combination is
ranked by the optimality of solutions found.  The algorithm reaches
    acceptable results and finds input sequences with largest elements
    on the order of $10^{-3}$.  Lastly, an illustration of the complexity
    of the fitness space is produced by brute-force sampling period-2
    inputs and plotting a fitness map of their stabilizing effect on the
    network.

\vfill
\textbf{Thesis Committee}

\textbf{Adviser:}\hspace{14pt}Dr. Roger Gaborski \hfill \makebox[3in]{\hrulefill}

\vspace{10pt}

\textbf{Observer:}\hspace{7pt}Dr. Peter Anderson \hfill \makebox[3in]{\hrulefill}

\vspace{10pt}

\textbf{Reader:}\hspace{18pt}Thomas Borelli \hfill \makebox[3in]{\hrulefill} 


\signature{Dr. Roger Gaborski}

\end{letter}
\end{document}
